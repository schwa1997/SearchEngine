\documentclass{ceurart}

\usepackage{acronym}
\usepackage{subcaption}
\usepackage{natbib}
\usepackage{enumitem}
\UseRawInputEncoding

\renewcommand{\acffont}[1]{\textsl{#1}}

%%
%% end of the preamble, start of the body of the document source.
\begin{document}

%%
%% Rights management information.
%% CC-BY is default license.
\copyrightyear{2023}
\copyrightclause{Copyright for this paper by its authors.\\
  Use permitted under Creative Commons License Attribution 4.0
  International (CC BY 4.0).}

%%
%% This command is for the conference information
\conference{``Search Engines'', course at the master degree in ``Computer Engineering'', Department of Information Engineering, and at the master degree in ``Data Science'', Department of Mathematics ``Tullio Levi-Civita'', University of Padua, Italy. Academic Year 2022/2023}

%%
%% The "title" command
\title{SEUPD@CLEF Task 1: Information retrieval for English and French
documents: Team JIHUMING}

%%
%% The "author" command and its associated commands are used to define
%% the authors and their affiliations.
\author[1]{Jes\'us Moncada-Ram\'irez}[%
email=jesus.moncadaramirez@studenti.unipd.it
]

\author[1]{Isil Atabek}[%
email=isil.atabek@studenti.unipd.it
]

\author[1]{Huimin Chen}[%
email=huimin.chen@studenti.unipd.it
]

\author[1]{Michele	Canale}[%
email=michele.canale.1@studenti.unipd.it
]

\author[1]{Nicol\`o Santini}[%
email=nicolo.santini.1@studenti.unipd.it
]

\author[1]{Giovanni Zago}[%
email=giovanni.zago.3@studenti.unipd.it
]


%%
%% The abstract is a short summary of the work to be presented in the
%% article.
\begin{abstract}
  % Presentation
  Our group will propose a novel search engine for the Longitudinal Evaluation of Model Performance (LongEval) task at
  CLEF 2023~\cite{LongEval};
  it will also be the final work of the subject Search Engines at the University of Padova.
  % Objective
  Our system focuses on the short-term and long-term temporal persistence of the systems' performance, for a collection
  of both English and French documents.
  % Approach
  Our approach involves analyzing both English and French versions of the documents using whitespace tokenization,
  stopword removal and stemming.
  We generate character N-grams to identify recurring word structures (as prefixes or suffixes) repeated over documents.
  We also use query expansion with synonyms (in English) and some Natural Language Processing (NLP) techniques as Named
  Entity Recognition (NER) to further refine our system.
  The similarity function utilized in our approach is BM25.
  Our system was developed in Java and primarily utilized the Lucene library.
  % Experiments and results
  After extensive experiments on these techniques, we came up with five systems that have produced the best results in
  terms of MAP and NDCG scores.
\end{abstract}

%%
%% Keywords. The author(s) should pick words that accurately describe
%% the work being presented. Separate the keywords with commas.
\begin{keywords}
  CLEF 2023 \sep
  Information retrieval \sep
  LongEval \sep
  English \sep
  French \sep
  Search Engines
\end{keywords}

%%
%% This command processes the author and affiliation and title
%% information and builds the first part of the formatted document.
\maketitle

\section{Introduction}\label{sec:introduction}
% Our group and what are we doing
This report aims at providing a brief explanation of the Information Retrieval system built as a team project during the
Search Engine course 22/23 of the master's degree in Computer Engineering and Data Science at the University of Padua,
Italy.
As a group in this subject, we are participating in the 2023 CLEF LongEval: Longitudinal Evaluation of Model
Performance~\cite{LongEval}.
This annual evaluation campaign focuses on the longitudinal evaluation of model performance in information retrieval and
natural language processing.\\

% The LongEval corpus
The LongEval collection~\cite{traindata} relies on a large set of data provided by Qwant (a commercial privacy-focused
search engine that was launched in France in 2013).
Their idea regarding the dataset (collected in June 2022) was to reflect changes of the Web across time, providing
evolving document and query sets.
The training collection consists of 672 \textbf{queries}, 98 held-out queries, and 9656 evaluation assignments.
The \textbf{documents} were chosen based on queries using the Qwant click model, in addition to random selection from
the Qwant index.
The training queries are categorized into twenty \textbf{topics}, such as: car-related, antivirus-related,
employment-related, energy-related, recipe-related, etc.
In addition to the original French version, the collection also includes English translations of the documents and
queries using the CUBBITT~\cite{CUBBITT} system.\\

% Organization
The paper is organized as follows:
Section~\ref{sec:methodology} briefly describes our approach;
Section~\ref{sec:architecture} describes our code in detail;
Section~\ref{sec:setup} explains our experimental setup;
Section~\ref{sec:results} discusses our main findings; finally,
Section~\ref{sec:conclusion} draws some conclusions and outlooks for future work.

\section{Methodology}\label{sec:methodology}

Our final search engine can be divided into the following parts: parsing of the documents and queries, indexing,
text processing (analyzers), and run generation (effective search).\\

Document parsing was performed using the JSON version of the documents.
On the other hand, query parsing was based on an XML parser.
See Section~\ref{subsec:parsing} for more details.\\

In the index, we decided to include four fields: (1) the (processed) English version of the documents, (2) the
(processed)French version, (3) character N-grams of both versions concatenated, and (4) some NER information extracted
from the French (original) version.
As similarity function we have used BM25~\cite{BM25} as it takes into account both term frequency and document length.
See Section~\ref{subsec:index} for more details.\\

The text added to the fields must first be processed, for this we have developed four different analyzers.
The English analyzer is based on whitespace tokenization, breaking of words and numbers based on special characters,
lowercasing, applying the Terrier~\cite{OunisEtAl2006} stopword list, query expansion with synonyms based on
the WordNet synonym map~\cite{wordnet}, and stemming.
The French analyzer is based on whitespace tokenization, breaking of words and numbers based on special characters,
lowercasing, applying a French stopword list~\cite{stopword_french} and stemming.
To generate the character N-grams we consider only the letters of the documents (i.e. we discard numbers and punctuation).
To perform NER we apply NLP techniques based on Apache OpenNLP~\cite{ApacheOpenNLP}) to the original (French) version of
the documents.
Specifically, we used NER applied to locations, person names and organizations.
See Section~\ref{subsec:analyzer} for more details.\\

We conducted some experiments to generate the runs, i.e., we have tried different combinations of the explained
techniques.
Thus, our searcher will always use BM25~\cite{BM25}, but the rest of characteristics depend on the run it is generating.
See Section~\ref{sec:setup} for more details.

\section{System Architecture}\label{sec:architecture}

In this section, we address the technical aspects of how our system was developed following the structure (in packages)
of the repository~\cite{jihuming}.

\subsection{Parsing}\label{subsec:parsing}
To generate an index from the provided documents, we parse them by extracting their text into Java data structures.
Our parser package is based on the JSON version of the documents, allowing easy manipulation and querying using various tools 
and libraries. We implemented a streaming parser using the \texttt{Gson} library in Java.\\

The whole parser is made up of the following classes:
\begin{itemize}
    \item \texttt{DocumentParser}:  An abstract class that represents a streaming parser and implements \texttt{Iterator} 
    and \text{Iterable}.
    \item \texttt{JsonDocument}: a Java POJO for the deserialization of JSON documents.
    \item \texttt{ParsedDocument}: Represents a parsed document, containing an identifier and a body.
    \item \texttt{LongEvalParser}: Implements the \texttt{DocumentParser} class and handles the streaming logic. 
    Objects of this class can be used as iterators to yield parsed documents
\end{itemize}

\subsection{Analyzer}\label{subsec:analyzer}
To process the parsed document text, we developed our own Lucene analyzers. Each analyzer follows a typical workflow 
involving a \texttt{Tokenizer} and a list of \texttt{TokenFilter} for a \texttt{TokenStream}
To process the already parsed documents' text, we have implemented our own Lucene analyzers.
All of them follow the typical workflow: use a \texttt{Tokenizer} and a list of \texttt{TokenFilter} to a
\texttt{TokenStream}.\\

The project's final version creates an index with four fields for each document, requiring four different analyzers. 
The \texttt{AnalyzerUtil}described below utilize functionalities from the AnalyzerUtil helper class developed by Nicola Ferro.\\

\subsubsection{English body field}
The processing applied to the English version of the documents (using the \texttt{EnglishAnalyzer} class) includes:
\begin{enumerate}
    \item Tokenize based on whitespaces.
    \item Eliminate some strange characters found in the documents. It is unlikely that a user would perform a query 
    including these characters.
    \item Removal of punctuation marks at the beginning and end of words since whitespace tokenization is used.
    \item Application of the \texttt{WordDelimiterGraphFilter} Lucene filter to split words into subwords based on case, 
    divide numbers, concatenate numbers with special characters, and remove English possessive trailing "s.".
    \item Lowercase all the tokens.
    \item Apply the Terrier~\cite{OunisEtAl2006} stopword list.
    \item Apply query expansion with synonyms using the \texttt{SynonymTokenFilter} from Lucene, based on the WordNet 
    synonym map~\cite{wordnet}. 
    \item Apply minimal stemming using the \texttt{EnglishMinimalStemFilter}from Lucene.  
    \item Removal of empty tokens left by previous filters using a custom \texttt{EmptyTokenFilter}
\end{enumerate}

\subsubsection{French body field}
The processing of French documents (in the class \texttt{FrenchAnalyzer} is identical to the processing of English 
documents in the first 5 points (excluding the English possessives’ removal in 4.d). For this point on, we apply: 
\begin{enumerate}[start=6]
    \item Apply a French stopword list~\cite{stopword_french}.
    \item Apply a minimal stemming process (in French) using \texttt{FrenchMinimalStemFilter} from Lucene.
    \item Removal of empty tokens (\texttt{EmptyTokenFilter}).
\end{enumerate}

\subsubsection{Character N-grams}
Character N-grams are created using the \texttt{NGramAnalyzer} class, which performs the following operations:
\begin{enumerate}
    \item Tokenize based on whitespaces.
    \item Lowercase all the tokens.
    \item Removal of all characters except letters (including French accent letters).
    \item Removal of empty tokens \texttt{EmptyTokenFilter}.
    \item Generate character N-grams using \texttt{NGramTokenFilter} from Lucene.
\end{enumerate}
The value of N has not been fixed in order to allow for the generation of different experiments.
See Section~\ref{sec:setup} for more details.

\subsubsection{NER extracted information}
The NER information has been extracted using the Apache OpenNLP~\cite{ApacheOpenNLP} library.
As Lucene does not include these functionalities directly, we have used a modified version of a token filter developed
by Nicola Ferro based on the mentioned library, (\texttt{OpenNLPNERFilter}).\\

The processing of the tokens in this analyzer (\texttt{NERAnalyzer}) is the following:
\begin{enumerate}
    \item Tokenization using the \texttt{StandardTokenizer} from Lucene.
    \item NER tagging using a model for locations.
    \item NER tagging using a model for person names.
    \item NER tagging using a model.
\end{enumerate}

\subsection{Index}\label{subsec:index}
Initially, we developed a \texttt{DirectoryIndexer} to handle single-language documents (English or French). 
However, when considering both versions, we deprecated it in favor of the final \texttt{MultilingualDirectoryIndexer}.\\

The \texttt{MultilingualDirectoryIndexer} is used for indexing multilingual documents and obtaining basic vocabulary statistics. 
To create an instance, parameters such as document directory paths, index directory path, expected document count, and custom 
analyzers (EnglishAnalyzer, FrenchAnalyzer, NGramAnalyzer, and NERAnalyzer) are required. 
Additionally, the chosen similarity function (BM25) and the RAM buffer size for indexing must be specified.\\

During indexing, the \texttt{MultilingualDirectoryIndexer} reads documents from the English and French directories, 
processes them with the specified analyzers, and creates an inverted index. Both directories must contain the same 
number of files and documents with matching IDs. Each iteration combines the English and French versions of the same 
document into a single Lucene document in the index.\\

After indexing, we utilize a method to print vocabulary statistics, including unique terms, total terms, and frequency 
lists for English and French. This provides a useful overview for analysis and optimization of the search system. 
The indexer also estimates the remaining time required for indexing, addressing the time-consuming nature of the process.\\

\subsection{Search}\label{subsec:search}
The \texttt{Searcher} class in the search package performs effective searches by applying the specified analyzers 
to the query title and matching it with the corresponding index fields. The \texttt{QueryParser} class from Lucene is 
utilized for this process. The search is conducted using the BM25 similarity function. Users can specify the index path, 
topics file path, number of expected topics, run descriptor, and the maximum number of documents to retrieve (1000). 
A user-friendly menu allows the selection of desired runs and distinguishes between train and test data.\\

\subsection{Topic}\label{subsec:topic}

To read queries (in TREC format) we developed our own LongEval topic reader (\texttt{LongEvalTopicReader}) in the topic package. 
It consists of the LongEvalTopic and LongEvalTopicReader classes. 
The \texttt{LongEvalTopic} represents each query with a number (\texttt{<num>}) and a title (\texttt{<title>}), 
serving as the equivalent of \texttt{QualityQuery} in \texttt{TrecTopicsReader}. 
The \texttt{LongEvalTopicReader} parses the query file as an XML file using the Java XML library.

\section{Experimental Setup}\label{sec:setup}

Our work was initiated based on the experimental setups outlined below.
\begin{itemize}
	\item Evaluation measures: MAP (Mean Average Precision) and NDCG (Normalized Discounted Cumulative Gain) scores.
	\item~\citep[Repository]{jihuming}.
	\item During the development and the experimentation, personal computers were used.
	\item Java JDK version 17, Apache version 2, Lucene version 9.5, and Maven.
\end{itemize}
In order to do different run experiments our team has created several indexes from the provided collection during the development of the final version of the project. In other words, the first created indexes only include several characteristics explained in this report, while the last indexes correspond to the final version of the project.\\
All the created indexes are \textbf{multilingual}, which allows us to take full advantage of the (bilingual) data collection. Additionally, we did some experiments with character N-grams generating different versions of indexes with 3-grams, 4-grams and 5-grams. Our motivation for experimenting with this was to compare how the size of different character N-grams affect to the effectiveness of our system. 3-grams are able to collecting more specific information about our documents, while 4-grams and 5-grams are more open to the context. An additional functionality of some indexes is query expansion, but as commented, this is only applied to the English body. One index uses Named Entity Recognition which provides not only the search for keywords but also identifying and extracting specific named entities.\\
In order to do different run experiments our team has created several indexes from the provided collection during the
development of the final version of the project.
In other words, the first created indexes only include several of the characteristics explained in this report, while
the last indexes correspond to the final version of the project.\\
All the created indexes are multilingual, which allows us to take full advantage of the (bilingual) data collection.
Additionally, we did some experiments with character 3-grams, 4-grams and 5-grams.
Our motivation for experimenting with this was to compare how the size of different character N-grams affect to the
effectiveness of our system. 3-grams are able to collect more specific information, while 4-grams and 5-grams allow considering bigger structures with more context and information. An additional functionality of some indexes is query expansion, but as commented, this is only applied to the English body. Finally, we created indexes with NER, which provides not only the search for keywords but also identifying and extracting specific named entities.\\

The subsequent indexes are:
\begin{itemize}
	\item \texttt{2023\_04\_24\_multilingual\_3gram}: both languages of documents, using character 3-grams.
	\item \texttt{2023\_04\_29\_multilingual\_3gram\_synonym}: both languages, character 3-grams, (English) query expansion with synonyms.
	\item \texttt{2023\_05\_01\_multilingual\_4gram\_synonym}: both languages, character 4-grams, (English) query expansion with synonyms.
	\item \texttt{2023\_05\_01\_multilingual\_5gram\_synonym}: both languages, character 5-grams, (English) query expansion with synonyms.
	\item \texttt{2023\_05\_05\_multilingual\_4gram\_synonym\_ner}: both languages, character 4-grams, (English) query expansion with synonyms, NER techniques.
\end{itemize}

The indexes also can be found in the following
\href{https://drive.google.com/drive/folders/1CK_kLeZ5Us3VJe8hiG1vhwPrDs94cLvU?usp=share_link}{Google Drive folder}.\\

After creating indexes, we were able to conduct multiple runs to evaluate the effectiveness of our system.
These runs not only experiment with some of the techniques specified here, but also consider different versions (English
or French version) of the queries.
With them we can compare and analyze different aspects of our system's performance, such as precision and recall.
We then computed the MAP and NDCG scores for each run, which allowed us to further evaluate the performance of our 
system.
The results will be commented in the Section~\ref{sec:results}.
The runs are the following:
\begin{itemize}
	\item \texttt{seupd2223-JIHUMING-01\_en\_en}: English topics; using English body field.
	\item \texttt{seupd2223-JIHUMING-02\_en\_en\_3gram}: English topics; using English body field and 3-gram field.
	\item \texttt{seupd2223-JIHUMING-03\_en\_en\_4gram}: English topics; using English body field and 4-gram field.
	\item \texttt{seupd2223-JIHUMING-04\_en\_en\_5gram}: English topics; using English body field and 5-gram field.
	\item \texttt{seupd2223-JIHUMING-05\_en\_en\_fr\_5gram}: English topics; using English and French body fileds and 5-gram field.
	\item \texttt{seupd2223-JIHUMING-06\_en\_en\_4gram\_ner}: English topics; using English body field, 4-gram field and NER technique.
	\item \texttt{seupd2223-JIHUMING-07\_fr\_fr}: French topics; using French body field.
	\item \texttt{seupd2223-JIHUMING-08\_fr\_fr\_3gram}: French topics; using French body field and 3-gram field.
	\item \texttt{seupd2223-JIHUMING-09\_fr\_fr\_4gram}: French topics; using French body field and 4-gram field.
	\item \texttt{seupd2223-JIHUMING-10\_fr\_fr\_5gram}: French topics; using French body field and 5-gram field.
	\item \texttt{seupd2223-JIHUMING-11\_fr\_en\_fr\_5gram}: French topics; using English and French body fields and 5-gram field.
	\item \texttt{seupd2223-JIHUMING-12\_fr\_fr\_4gram\_ner}: French topics; using French body field, 4-gram field and NER technique.
\end{itemize}

The process of creating the indexes typically took around 1 hour, with the exception of the indexes that included NER,
which took approximately 16 hours.
On the other hand, generating the runs was a much quicker process, taking consistently less than a minute and a half to
complete.

\section{Results and Discussion}
\label{sec:results}
\begin{table}[h!]
    \begin{center}
        \caption{MAP and NCDG scores for all runs}
        \label{tab:all_scores}
        \begin{tabular}{|c|c||c|c|} 
            \hline
            \textbf{Index} & \textbf{Run} & \textbf{MAP Score} & \textbf{NCDG Score}\\
            \hline\hline
            01 & en\_en & \cellcolor{red!30!white}0.0700 & \cellcolor{red!30!white}0.1614 \\
            \hline
            02 & en\_en\_3gram & 0.0704 & 0.1661 \\
            \hline
            03 & en\_en\_4gram & 0.0874 & 0.2025 \\
            \hline
            04 & en\_en\_5gram & 0.1028 & 0.2288 \\
            \hline
            05 & en\_en\_fr\_5gram & \cellcolor{red!60!white}0.0669 & \cellcolor{red!60!white}0.1525 \\
            \hline
            06 & en\_en\_4gram\_ner & \cellcolor{red}0.0360 & \cellcolor{red}0.1098 \\
            \hline
            07 & fr\_fr & 0.1656 & 0.3135 \\
            \hline
            08 & fr\_fr\_3gram & \cellcolor{green!30!white}0.1698 & \cellcolor{green!30!white}0.3208 \\
            \hline
            09 & fr\_fr\_4gram & \cellcolor{green!60!white}0.1737 & \cellcolor{green!60!white}0.3269 \\
            \hline
            10 & fr\_fr\_5gram & \cellcolor{green}0.1748 & \cellcolor{green}0.3285 \\
            \hline
            11 & fr\_en\_fr\_5gram & 0.1288 & 0.2797 \\
            \hline
            12 & fr\_fr\_4gram\_ner & 0.1362 & 0.2881 \\
            \hline
        \end{tabular}
    \end{center}
\end{table}

\begin{figure}[h!]
	\centering
	\includegraphics[width=0.6\textwidth]{figure/allScores.png}
	\caption{All scores sorted by MAP score}
	\label{fig:sorted_scores}
\end{figure}
The analysis shows that the highest MAP score (0.1748) is achieved by \texttt{fr\_fr\_5gram}, followed by \texttt{fr\_fr\_4gram} (0.1737) and \texttt{fr\_fr\_3gram} (0.1698), while the lowest MAP score (0.0360) is obtained by \texttt{en\_en\_4gram\_ner}.
Similarly, the highest NDCG score (0.3208) belongs to \texttt{fr\_fr\_4gram\_ner}, followed by \texttt{fr\_fr\_5gram} (0.3285) and \texttt{fr\_fr\_4gram} (0.3269), whereas the lowest NDCG score (0.1098) corresponds to \texttt{en\_en\_4gram\_ner}.\\
Results suggest that French queries perform better than their English counterparts, possibly due to the training data's French origin and later translation into English. Moreover, the IR system's effectiveness generally increases with a larger N-gram size, as indicated by the higher scores
of \texttt{en\_en\_5gram} and \texttt{fr\_fr\_5gram}. Conversely, the inclusion of NER in the indexing process seems to have a negative impact on the scores, as shown by the lower scores of \texttt{en\_en\_4gram\_ner} and \texttt{fr\_fr\_4gram\_ner}.
The use of query expansion with synonyms in English does not seem to improve the search results to any great extent.\\
Here we can see a chart ranking of the five best scores, they are the runs that have been presented at CLEF:
\begin{figure}[h!]
	\centering
	\includegraphics[width=0.6\textwidth]{figure/bestScores.png}
	\caption{Best MAP and NDCG scores}
	\label{fig:best_scores}
\end{figure}
It's interesting to notice that the cross-language approaches (\texttt{en\_en\_fr\_5gram} and \texttt{fr\_en\_fr\_5gram}) are out of the five bests systems.
It turns out that searching for English words in French documents and vice versa messes up the search, lowering the score.
Another interesting aspect is that the worst-performing index is the one with named entity recognition in English (\texttt{en\_en\_4gram\_ner}): it combines translated queries and NER, which appears to be the two worst-performing approaches.\\
In general, we focus more on trying multiple approaches, this is why our score has such a big space for improvement.
As already said, French queries with bigger N-gram sizes perform better.
Instead of relying on single-word matches, the queries could take place with more context, resulting in better search
results.\\
Following the competition workflow, we created the indexes based on the test data and re-executed the top five
runs (see Figure~\ref{fig:best_scores}). These runs will be the ones delivered to CLEF.

\section{Conclusions and Future Work}
\label{sec:conclusion}

In summary, the IR systems developed in this study followed the Parsing-Analyzer-Index-Search-Topic paradigm and
utilized different methodologies, among which the following stand out: processing of English documents based on
whitespace tokenization, the TERRIER stopword list, query expansion and stemming;
processing of French documents based on whitespace tokenization, a stopword list and stemming;
character N-grams of both versions concatenated;
and NER information extraction using NLP techniques.\\
We evaluated the performance of the 12 systems we developed by measuring the effectiveness of runs on the training data, comprising both French and translated English queries and documents.
To assess the quality of these runs, we used the MAP and NDCG scores calculated by \texttt{trec\_eval}.
Among these systems, five models performed the best, namely \texttt{fr\_fr\_5gram}, \texttt{fr\_fr\_4gram}, \texttt{fr\_fr\_3gram}, \texttt{fr\_fr}, and \texttt{fr\_fr\_4gram\_ner}, listed in order of their scores
from highest to lowest for both MAP and NDCG.\\
In terms of future work, there are several areas that could be explored to improve the effectiveness of the developed IR systems.
Firstly, we could improve indexing methodologies, such as increasing the value of N of N-gram, as we have commented on in Section~\ref{sec:results}.
Secondly, we could explore better NLP techniques to improve the accuracy of the IR systems, as NER turns out not to be very effective. \\
One last possible future work could be a machine-learning based IR system. Using training data, we could train a model to predict the best N for N-grams, the best analyzer, and the best index for a given query and document.
This would be a more dynamic approach to IR systems, as it would be able to adapt to different types of queries and documents.

%This command is used to dont compile all code tile \fi
\iffalse
\section{Misc [TO BE REMOVED]}

\subsection{Tex Files}

Put your \LaTeX files into the \texttt{section} folder as shown in the examples above.

\subsection{Figures}

Put your figures into the \texttt{figure} folder and put the caption under the figure. Example of reference to Figure~\ref{fig:sample-figure}.

\begin{figure}
  \centering
  \includegraphics[width=0.8\linewidth]{figure/sample.pdf}
  \caption{1907 Franklin Model D roadster. Photograph by Harris \& Ewing, Inc. [Public domain], via Wikimedia Commons. (\url{https://goo.gl/VLCRBB}).}
  \label{fig:sample-figure}
\end{figure}

\subsection{Tables}

Put the caption above the table. Example of reference to Table~\ref{tab:sample-table}.

\begin{table}
  \caption{Frequency of Special Characters}
  \label{tab:sample-table}
  \centering
  \begin{tabular}{|c|c|l|}
    \toprule
    Non-English or Math&Frequency&Comments\\
    \midrule
    \O & 1 in 1,000& For Swedish names\\
    $\pi$ & 1 in 5& Common in math\\
    \$ & 4 in 5 & Used in business\\
    $\Psi^2_1$ & 1 in 40,000& Unexplained usage\\
  \bottomrule
\end{tabular}
\end{table}

See the \texttt{booktab} packaged documentation for further options.

\subsection{Bibliography}

Example of citations:
\begin{itemize}
	\item name: \citet{Salton1968}
	\item parenthesis: \citep{Salton1968}
\end{itemize}

An initial list of references is provided in the files \texttt{bibliography.bib} and \texttt{proceedings.bib} that you can expand.

See the \texttt{natbib} packaged documentation for further options.

\subsection{Acronyms}

Use the 

\begin{verbatim}
 \ac{acronym}
 \end{verbatim}

command to insert acronyms, eg. \ac{AP}. The command will expand the acronym the first time it is used.

An initial list of acronyms is provided in the file \texttt{acronyms.tex} that you can expand.

See the \texttt{acronym} packaged documentation for further options.
\fi

%% Define the bibliography file to be used
\bibliography{bibliography,proceedings}

\input{acronyms}


\end{document}
