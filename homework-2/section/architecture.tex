\section{System Architecture}\label{sec:architecture}

In this section, we address the technical aspects of how our system was developed following the structure (in packages)
of the repository~\cite{jihuming}.

\subsection{Parsing}\label{subsec:parsing}
To generate an index from the provided documents, we parse them by extracting their text into Java data structures.
Our parser package is based on the JSON version of the documents, allowing easy manipulation and querying using various tools 
and libraries. We implemented a streaming parser using the \texttt{Gson} library in Java.\\

The whole parser is made up of the following classes:
\begin{itemize}
    \item \texttt{DocumentParser}:  An abstract class that represents a streaming parser and implements \texttt{Iterator} 
    and \text{Iterable}.
    \item \texttt{JsonDocument}: a Java POJO for the deserialization of JSON documents.
    \item \texttt{ParsedDocument}: Represents a parsed document, containing an identifier and a body.
    \item \texttt{LongEvalParser}: Implements the \texttt{DocumentParser} class and handles the streaming logic. 
    Objects of this class can be used as iterators to yield parsed documents
\end{itemize}

\subsection{Analyzer}\label{subsec:analyzer}
To process the parsed document text, we developed our own Lucene analyzers. Each analyzer follows a typical workflow 
involving a \texttt{Tokenizer} and a list of \texttt{TokenFilter} for a \texttt{TokenStream}
To process the already parsed documents' text, we have implemented our own Lucene analyzers.
All of them follow the typical workflow: use a \texttt{Tokenizer} and a list of \texttt{TokenFilter} to a
\texttt{TokenStream}.\\

The project's final version creates an index with four fields for each document, requiring four different analyzers. 
The \texttt{AnalyzerUtil}described below utilize functionalities from the AnalyzerUtil helper class developed by Nicola Ferro.\\

\subsubsection{English body field}
The processing applied to the English version of the documents (using the \texttt{EnglishAnalyzer} class) includes:
\begin{enumerate}
    \item Tokenize based on whitespaces.
    \item Eliminate some strange characters found in the documents. It is unlikely that a user would perform a query 
    including these characters.
    \item Removal of punctuation marks at the beginning and end of words since whitespace tokenization is used.
    \item Application of the \texttt{WordDelimiterGraphFilter} Lucene filter to split words into subwords based on case, 
    divide numbers, concatenate numbers with special characters, and remove English possessive trailing "s.".
    \item Lowercase all the tokens.
    \item Apply the Terrier~\cite{OunisEtAl2006} stopword list.
    \item Apply query expansion with synonyms using the \texttt{SynonymTokenFilter} from Lucene, based on the WordNet 
    synonym map~\cite{wordnet}. 
    \item Apply minimal stemming using the \texttt{EnglishMinimalStemFilter}from Lucene.  
    \item Removal of empty tokens left by previous filters using a custom \texttt{EmptyTokenFilter}
\end{enumerate}

\subsubsection{French body field}
The processing of French documents (in the class \texttt{FrenchAnalyzer} is identical to the processing of English 
documents in the first 5 points (excluding the English possessives’ removal in 4.d). For this point on, we apply: 
\begin{enumerate}[start=6]
    \item Apply a French stopword list~\cite{stopword_french}.
    \item Apply a minimal stemming process (in French) using \texttt{FrenchMinimalStemFilter} from Lucene.
    \item Removal of empty tokens (\texttt{EmptyTokenFilter}).
\end{enumerate}

\subsubsection{Character N-grams}
Character N-grams are created using the \texttt{NGramAnalyzer} class, which performs the following operations:
\begin{enumerate}
    \item Tokenize based on whitespaces.
    \item Lowercase all the tokens.
    \item Removal of all characters except letters (including French accent letters).
    \item Removal of empty tokens \texttt{EmptyTokenFilter}.
    \item Generate character N-grams using \texttt{NGramTokenFilter} from Lucene.
\end{enumerate}
The value of N has not been fixed in order to allow for the generation of different experiments.
See Section~\ref{sec:setup} for more details.

\subsubsection{NER extracted information}
The NER information has been extracted using the Apache OpenNLP~\cite{ApacheOpenNLP} library.
As Lucene does not include these functionalities directly, we have used a modified version of a token filter developed
by Nicola Ferro based on the mentioned library, (\texttt{OpenNLPNERFilter}).\\

The processing of the tokens in this analyzer (\texttt{NERAnalyzer}) is the following:
\begin{enumerate}
    \item Tokenization using the \texttt{StandardTokenizer} from Lucene.
    \item NER tagging using a model for locations.
    \item NER tagging using a model for person names.
    \item NER tagging using a model.
\end{enumerate}

\subsection{Index}\label{subsec:index}
Initially, we developed a \texttt{DirectoryIndexer} to handle single-language documents (English or French). 
However, when considering both versions, we deprecated it in favor of the final \texttt{MultilingualDirectoryIndexer}.\\

The \texttt{MultilingualDirectoryIndexer} is used for indexing multilingual documents and obtaining basic vocabulary statistics. 
To create an instance, parameters such as document directory paths, index directory path, expected document count, and custom 
analyzers (EnglishAnalyzer, FrenchAnalyzer, NGramAnalyzer, and NERAnalyzer) are required. 
Additionally, the chosen similarity function (BM25) and the RAM buffer size for indexing must be specified.\\

During indexing, the \texttt{MultilingualDirectoryIndexer} reads documents from the English and French directories, 
processes them with the specified analyzers, and creates an inverted index. Both directories must contain the same 
number of files and documents with matching IDs. Each iteration combines the English and French versions of the same 
document into a single Lucene document in the index.\\

After indexing, we utilize a method to print vocabulary statistics, including unique terms, total terms, and frequency 
lists for English and French. This provides a useful overview for analysis and optimization of the search system. 
The indexer also estimates the remaining time required for indexing, addressing the time-consuming nature of the process.\\

\subsection{Search}\label{subsec:search}
The \texttt{Searcher} class in the search package performs effective searches by applying the specified analyzers 
to the query title and matching it with the corresponding index fields. The \texttt{QueryParser} class from Lucene is 
utilized for this process. The search is conducted using the BM25 similarity function. Users can specify the index path, 
topics file path, number of expected topics, run descriptor, and the maximum number of documents to retrieve (1000). 
A user-friendly menu allows the selection of desired runs and distinguishes between train and test data.\\

\subsection{Topic}\label{subsec:topic}

To read queries (in TREC format) we developed our own LongEval topic reader (\texttt{LongEvalTopicReader}) in the topic package. 
It consists of the LongEvalTopic and LongEvalTopicReader classes. 
The \texttt{LongEvalTopic} represents each query with a number (\texttt{<num>}) and a title (\texttt{<title>}), 
serving as the equivalent of \texttt{QualityQuery} in \texttt{TrecTopicsReader}. 
The \texttt{LongEvalTopicReader} parses the query file as an XML file using the Java XML library.